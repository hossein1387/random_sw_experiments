\section{Hamming Distance}
\subsection{Theory}
Hamming distance of two integers, each 32 bits, is the number of changes that are necessary to change one integer to another. In other words, if A and B are two 32 bits integer, the hamming distance between the two, denoted by $hamdist(A,B)$ is defined as follows:
\begin{equation}\label{hammingdist}
		hamdist(A,B) = \sum_{i=0}^{31}\left[A(i) \neq B(i)\right],
\end{equation}
where $A(i)$ represents the $i^{th}$ bit of A. Hamming distance is bounded between 0 (in the case that $\forall i, A(i)==B(i)$) and 32 (in the case that $\forall i, A(i)\neq B(i)$). Hamming distance has the following properties:
\begin{itemize}

\item{}
Hamming distance is a symmetric operator as the hamming distance of A and B is the same as hamming distance of B and A:
\begin{equation}\label{symmetry}
		hamdist(A, B) == hamdist(B,A)
\end{equation}

\item{}
The hamming distance of a 32 bit integer is bounded between 0 and 32. In general, the hamming distance of two n-bit numbers is bounded between 0 and n:
\begin{equation}\label{bounded}
	\forall A,B \in \left[0, 2^n-1 \right], \mbox{where}\ A, B \in \mathbb{N}, 0 \leq hamdist(A, B) \leq n
\end{equation}

\item{}
The hamming distance of any integer with it self is zero.  

\begin{equation}\label{ZReflect}
		hamdist(A, A) == 0
\end{equation}
\end{itemize}

In the following section, we use the above-mentioned properties to compute the minimum hamming distance of an array of integers.

\subsection{Minimum Hamming Distance of an array of 32-bit integers}
The minimum hamming distance of an array is the minimum hamming distance of any two distinct integers of the array. The minimum hamming distance of array $A$ is defined as follows:
\begin{equation}\label{HammingDistArray}
	\begin{gathered}
         i, j, n \in \mathbb{N}\\
        \forall i \in \left[0, n\right]\\
        \forall j \in \left[0, n\right]\\
        d = min(hamdist(A(i), A(j)))
	\end{gathered}
\end{equation}

Here, we use the properties of hamming distance operator in order to further improve the algorithm.

\begin{equation}\label{HammingDistArrayOpt}
	\begin{gathered}
         i, j, n \in \mathbb{N}\\
        \forall i \in \left[0, n\right]\\
        \forall j \in \left[i+1, n\right]\\
        d = min(hamdist(A(i), A(j)))
	\end{gathered}
\end{equation}

The above algorithm reduces the number of iterations required for finding the minimum distance. We used \ref{symmetry} to avoid computing hamming distance of $hamdist(A(i), A(j))$ when $hamdist(A(j), A(i))$ has already been computed. We used \ref{ZReflect} to avoid computing hamming distance when $i==j$. These two improvements were achieved by starting $j$ from $i+1$. Since we are looking for minimum hamming distance, we used \ref{bounded} to exit the loop as soon as a minimum distance of 0 is computed. 

\subsection{Source Code And Results}
With the algorithm defined in the previous section, we have attempted to implement it using C++. The source code for this project can be founded in this github repository: \href{https://github.com/hossein1387/random_sw_experiments/tree/master/hamming_distance}{GitHub Repository For Min Hamming Distance}. 
\markdownInput{Hamming.md}
